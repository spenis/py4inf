% The contents of this file is 
% Copyright (c) 2009- Charles R. Severance, All Righs Reserved

\chapter{Πρόλογος}

\section*{Python Πάιθον για Πληροφορική: Αναδημιουργώντας ένα Ανοιχτό Βιβλίο}

Είναι κάτι αναμενόμενο για τους και τις ακαδημαϊκούς που διαρκώς τους προτρέπουν να
It is quite natural for academics who are continuously told to 
''εκδόσουν ή να ξεχαστούν'' να επιθυμούν τη δημιουργία κάτι νέου, από το μηδέν
``publish or perish'' to want to always create something from scratch
που θα λογίζεται ως το δικό τους μοναδικό δημιούργημα. Αυτό το βιβλίο είναι ένας 
that is their own fresh creation.   This book is an 
πειραματισμός στο πως να μην ξεκινάμε από το μηδέν, αλλά αντ' αυτού, μια ''μίξη''
experiment in not starting from scratch, but instead ``remixing''
του βιβλίου με τίτλο
the book titled
\emph{Σκέψου Πάϊθον: Πώς να σκέφτεσαι ως Think Python: How to Think Like
επιστήμονας ηλεκτρονικών υπολογιστών a Computer Scientist}
γραμμένο από τους Allen B. Downey, Jeff Elkner, και άλλους.
written by Allen B. Downey, Jeff Elkner, and others.

Τον Δεκέμβρη του 2009, καθώς προετοιμαζόμουν για τη διδαχή του  
In December of 2009, I was preparing to teach
μαθήματος SI502 - Προγραμματισμός Δικτύων, στο Πανεπιστήμιου του Μίτσιγκαν των ΗΠΑ
{\bf SI502 - Networked Programming} at the University of Michigan
για το πέμπτο ακαδημαϊκό εξάμηνο στην σειρά, αποφάσισα πως είχε έρθει η ώρα
for the fifth semester in a row and decided it was time
να γράψω ένα βιβλίο για την Πάιθον που θα εστίαζε στην εξερεύνηση των δεδομένων(data)
to write a Python textbook that focused on exploring data
παρά στην κατανόηση αλγόριθμων και επίπεδα αφαίρεσης.
instead of understanding algorithms and abstractions.
Ο στόχος μου δισάσκοντας το μάθημα SI502 είναι να εκπαιδεύσω όσες και όσους το παρακολουθούν, να αποκτήσουν δια βίου δεξιότητες χειρισμό δεδομένων 
My goal in SI502 is to teach people lifelong data handling 
χρησιμοποιόντας την γλώσσα προγραμματισμού Πάιθον. Ελάχιστοι από τους φοιτητές και φοιτήτριες μου 
skills using Python.  Few of my
σχεδίαζαν/επιθυμούσαν να γίνουν επαγγελματίες προγραμματιστές υπολογιστών.
students were planning to be professional 
Αντιθέτως, θα ακολουθούσαν μια καρίερα
computer programmers.  Instead, they
planned to be librarians, managers, lawyers, biologists, economists, etc., 
ως βιβλιοθηκονόμοι, μανατζερ, δικηγόροι, βιολόγοι, οικονομολόγοι, κλπ., 

που απλά έτυχε να επιθυμούν να χρησιμοποιούν επιδέξια την τεχνολογία στον τομέα επιλογή τους.
who happened to want to skillfully use technology in their chosen field.

Δεν μπόρεσα ποτέ να βρω το ιδανικό βιβλίο για να διδάξω στο μάθημα μου την αντικειμενοστρεφή Πάιθον,
I never seemed to find the perfect data-oriented Python
και γι'αυτό αποφάσια να το γράψω.
book for my course, so I set out 
Κατά τύχη, σε μια συνάντηση των καθηγητών του πανεπιστημίου, τρεις βδομάδες πριν ξεκινήσω να γράφω το βιβλίο μου από το μηδέν κατά τη διάρκεια των διακοπών, 
to write just such a book.  Luckily at a faculty meeting three weeks
before I was about to start my new book from scratch over 
the holiday break, 
ο διδάκτορ Atul Prakash μου έδειξε το βιβλίο \emph{Think Python} Σκέψου Πάιθον, το οποίο είχε   
Dr. Atul Prakash showed me the \emph{Think Python} book which he had
χρησιμοποιήσει για να διδάξει Πάιθον εκείνο το ακαδημαϊκό εξάμηνο.
used to teach his Python course that semester.  

Πρόκειται για ένα καλογραμμένο κείμενο της επιστήμης των υπολογιστών 
It is a well-written Computer Science text with a focus on 
με έμφαση σε σύντομες, άμεσες επεξηγήσεις και ευκολία εκμάθησης.
short, direct explanations and ease of learning.  

Η συνολική δομή του βιβλίου
The overall book structure
έχει αλλάξει με σκοπό να ασχολείται με προβλήματα ανάλυσης δεδομένων όσο πιο γρήγορα γίνεται
has been changed to get to doing data analysis problems as quickly as
μέσω μιας σειράς από τρέχοντα παραδείγματα και ασκήσεις
possible and have a series of running examples and exercises 
που καταπιάνονται με την ανάλυση δεδομένων από το μηδέν.
about data analysis from the very beginning.  

Τα κεφάλαια 2 έως 10 είναι παρόμοια με του Σκέψου Πάιθον,
Chapters 2--10 are similar to the \emph{Think Python} book,
αλλά έχουν γίνει μεγάλες αλλαγές. Τα νουμερικά παραδείγματα έχουν 
but there have been major changes. Number-oriented examples and
αντικατασταθεί με παραδείγματα δεδομένων. 
exercises have been replaced with data-oriented exercises.
Τα θέματα παρουσιάζονται με την σειρά που απαιτείται για να χτιστούν 
Topics are presented in the order needed to build increasingly
αυξανόμενα εκλεπτισμένες λύσεις ανάλυσης δεδομένων. Μερικά θέματα όπως 
sophisticated data analysis solutions. Some topics like {\tt try} and
έχουν μετακινηθεί παρακάτω και παρουσιάζονται ως μέρους του κεφαλαίου 
{\tt except} are pulled forward and presented as part of the chapter
περί υποθετικών. 
on conditionals.  Functions are given very light treatment until 
they are needed to handle program complexity rather than introduced 
as an early lesson in abstraction.  Nearly all user-defined functions
have been removed from the example code and exercises outside of Chapter 4.
The word ``recursion''\footnote{Except, of course, for this line.}
does not appear in the book at all.

In chapters 1 and 11--16, all of the material is brand new, focusing
on real-world uses and simple examples of Python for data analysis 
including regular expressions for searching and parsing, 
automating tasks on your computer, retrieving data across 
the network, scraping web pages for data, 
using web services, parsing XML and JSON data, and creating 
and using databases using Structured Query Language.

The ultimate goal of all of these changes is a shift from a 
Computer Science to an Informatics
focus is to only include topics into a first technology 
class that can be useful even if one chooses not to 
become a professional programmer.

Students who find this book interesting and want to further explore
should look at Allen B. Downey's \emph{Think Python} book.  Because there
is a lot of overlap between the two books,
students will quickly pick up skills in the additional
areas of technical programming and algorithmic thinking 
that are covered in \emph{Think Python}.
And given that the books have a similar writing style, they should be 
able to move quickly through \emph{Think Python} with a minimum of effort.

\index{Creative Commons License}
\index{CC-BY-SA}
\index{BY-SA}
As the copyright holder of \emph{Think Python},
Allen has given me permission to change the book's license 
on the material from his book that remains in this book
from the
GNU Free Documentation License 
to the more recent
Creative Commons Attribution --- Share Alike
license.
This follows a general shift in open documentation licenses moving 
from the GFDL to the CC-BY-SA (e.g., Wikipedia).
Using the CC-BY-SA license maintains the book's 
strong copyleft tradition while making it even more straightforward 
for new authors to reuse this material as they see fit.

I feel that this book serves an example of why open 
materials are so important to the future of education,
and want to thank Allen B. Downey and Cambridge University
Press for their forward-looking decision to make the book available
under an open copyright.   I hope they are pleased with the 
results of my efforts and I hope that you the reader are pleased with
\emph{our} collective efforts.

I would like to thank Allen B. Downey and Lauren Cowles for their help,
patience, and guidance in dealing with and resolving the copyright 
issues around this book.

Charles Severance\\
www.dr-chuck.com\\
Ann Arbor, MI, USA\\
September 9, 2013

Charles Severance is a 
Clinical Associate Professor 
at the University of Michigan School of Information.

\clearemptydoublepage

% TABLE OF CONTENTS
\begin{latexonly}

\tableofcontents

\clearemptydoublepage

\end{latexonly}

% START THE BOOK
\mainmatter

